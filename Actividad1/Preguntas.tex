\documentclass[12pt]{article}
\usepackage[utf8]{inputenc}
\usepackage[T1]{fontenc}
\usepackage[spanish]{babel}%caracteres en español
\title{\huge \textbf{\textsc{Actividad 1:\\ Sección de preguntas}}}%titulo en grande-negritas-versalitas
\author{J. Ernesto Torres Burruel}
\usepackage{graphicx}%para cargar imagenes
\graphicspath{{Imagenes/}}
\usepackage{wrapfig} %para acomodar figuras y que compartan espacio con texto
\usepackage{fancyhdr}
\pagestyle{fancy}
\fancyhf{}
\rfoot{Página \thepage}
\usepackage{enumerate}
\usepackage{cite}

% Enable SageTeX to run SageMath code right inside this LaTeX file.
% documentation: http://mirrors.ctan.org/macros/latex/contrib/sagetex/sagetexpackage.pdf
% \usepackage{sagetex}

\begin{document}

\begin{titlepage}%pagina inicial
\begin{figure}[h]
    \centering
    \includegraphics[width=8cm]{licfis}
    {\huge \maketitle}
    {\large Física Computacional\\Carlos Lizárraga Celaya}
\end{figure}

\thispagestyle{empty}%aún así no se pudo eliminar el número de página

\end{titlepage}
\pagenumbering{arabic} %solo con este comando se pudo eliminar el numero de página

\newpage

\section{Preguntas}

\begin{itemize}
\item \textbf{¿Cuál es tu primera impresión de uso de \LaTeX\ ?}

Para mi es muy diferente el manejo de \LaTeX\ comparado con el de otros procesadores de texto que son más dinámicos, tienen una mejor interfaz para el usuario y que poseen las mismas y/o más funciones que las que presenta \LaTeX\ .

Otra impresión que me generó fue la de tener que aprender a utilizar una gran cantidad de comandos y no todos funcionan de manera adecuada, por ejemplo, no pude eliminar el número de la primera página de mi actividad principal, la que es sobre la atmósfera. La dificultad para manejar los objetos (como imágenes) es también es un problema.

\item \textbf{¿Qué aspectos te gustaron más? }
Me gustó que (cuando funcionó antes de presentar bugs) la bibligrafía y el título es facil de hacer, así como tener un orden de los títulos(secciones) y subtitulos(subsecciones), que permite agregar fácilmente los índices, sin tener que estar recordando si pusiste el formato correcto como en otros editores de texto que debes de elegir también el tipo de texto que será, que muchas veces puedes olvidarlo o perder el orden entre ellos.

Me gustó que la justificación del texto ya está predefinida, sin embaro aprender comandos para tener otro tipo de presentación del texto, ya sea en itálicas, versalitas o negritas se tiene que tener precauciones al manejar estos formatos, pues los comandos pueden generar errores o bien cambiar todo el texto que proceda al comando.

\item \textbf{¿Qué no pudiste hacer en \LaTeX\ ?}
En este documento quise enumerar las preguntas, sin embargo sagemath me marcaba errores al usar el comando begin (eumerate) y su respectivo end. Algo tan sencillo como enumerar una lista y que no se realice de manera correcta es bastante decepcionante, aún cuando teniendo referencias de las guías de \LaTeX\ disponibles en la web. Tampoco pude realizar una tabla que mostrara datos dados por las páginas que realizan mediciones en la atmósfera, eso fue principalmente por falta de tiempo, ya que realizar una búsqueda de todos los comandos y escribirlos de manera que se compilen correctamente requeriría más tiempo. 

Otra deficiencia importante es que este editor de texto no tiene un corrector ortográfico ni mucho menos de sintaxis, tampoco permite de manera correcta la interferencia del diccionario de Google Chrome. Si bien este tipo de funciones de los editores de texto no permite una mejora en el cuidado de la escritura del autor, es bastante impráctico no tener un detector de errores, pues en español se utilizan muchas palabras acentuadas, las cuales es fácil perderlas de vista si se busca escribir rápidamente.

\item \textbf{ En tu experiencia, comparado con otros editores, ¿cómo se compara \LaTeX\ ? }
Latex es comparable con la cantidad de funciones que ofrece, sin embargo se queda corto con respecto a procesadores de texto como Word, pues este tiene una interfaz gráfica para las funciones y creaciones de elementos dentro del texto, así como también se tiene una gama más completa de edición de los formatos y fuentes del texto. Es bastante retrogrado pues los avances que se tienen en otros procesadores de texto revasan por mucho lo que ofrece \LaTeX\ simplemente con ofrecer todo lo de este editor y más, incluso hay procesadores de texto que trabajan de manera más dinámica que permiten la creación de un formato de \LaTeX\ (.tex). 


\item \textbf{¿Qué es lo que mas te llamó la atención en el desarrollo de esta actividad?}
Para mi fue conocer toda una nueva forma de realizar escritura de textos hechos en computadora, pues jamás había manejado un editor con estas características, también, aunque fue arduo, comencé a entender bastantes propiedades de .tex.

Con respecto a la investigación realizada me gustó conocer algo tan básico sobre la estructura de nuestro planeta (aunque solo fue sobre la atmósfera), ahora comprendo ciertas características de ella y me interesó la dimensión que tiene esta gran capa de la tierra, pues se considera que se extiende hasta casi la mitad de la distancia entre la tierra y la luna. Es bastante interesante conocer la gran cantidad de procesos físicos que se realizan en la atmósfera, desde los más básicos, hasta los más sofisticados como la ionización de elementos y fenómenos que suceden en las capas atmosféricas como la detencion de radicación y la descomposición de moléculas por causa de esto.


\item \textbf{¿Qué cambiarías en esta actividad?}
Me gustaría que fuera una actividad más enfocada al proceso de escritura en \LaTeX\ pues realizar la búsqueda de cada una de los comandos necesitados para todos los formatos de texto que queremos utiilizar es mucho más ardua y tardía, además de que se roba bastante tiempo de la investigación que se centra en la estructura de la atmósfera, tal vez esta serviría como una segunda actividad y la primera una sobre utillizar distintos comandos de \LaTeX\ para comenzar a introducirnos en el contexto de escritura, porque el inicio a \LaTeX\ lo sentí un poco forzado, pues yo era uno de muchos de quienes nunca habían utilizado este procesador de texto.

\item\textbf{ ¿Qué consideras que falta en esta actividad? }
Considero que es bastante completa, ya que la presión que ejerce para realizar la actividad motiva y a la vez frustra al introducirnos en el contexto de \LaTeX\ . Me gustaría que hubiera más fuentes disponibles de información sobre la investigación de la atmósfera

\item \textbf{¿Puedes compartir alguna referencia nueva que consideras util y no se haya contemplado?}
Una fuente que yo utilicé bastante para comprender más sobre la atmósfera fue la de \textit{University Corporation for Atmospheric Reseach}, que tiene un sitio web con información bastante completa sobre los temas de la atmósfera.

\item \textbf{¿Algún comentario adicional que desees compartir?}
Opino que el uso de \LaTeX\ para textos científicos está bastante sobrevalorada, pues, como he mencionado anteriormente, existen editores de texto mcuho más manejables y que presentan menos errores que \LaTeX\ ; este procesador de texto también debe de consumir bastante texto a los autores de artículos y libros ya que al necesitar un nuevo comando o al tener fallas en alguno de ellos debe de tomar mucho tiempo conseguir que funcione y perder tiempo en la creación de un documento no debería de ser una cuestión que afecte al desempeño de un científico, pues creo que interfiere mucho en el manejo del tiempo y sobre todo al tener errores poco visibles y también bastante complicados de resolver, cuando es una cuestión como un bug del procesador.

\end{itemize}

\newpage

\end{document}