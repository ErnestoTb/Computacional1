\documentclass[12pt]{article}
\usepackage[utf8]{inputenc}
\usepackage[T1]{fontenc}
\usepackage[spanish]{babel}%caracteres en español
\title{\huge \textbf{\textsc{La atmósfera}}}%titulo en grande-negritas-versalitas
\author{J. Ernesto Torres Burruel}
\usepackage{color}%para colores en el texto
\usepackage{graphicx}%para cargar imagenes
\graphicspath{{Imagenes/}}
\usepackage{wrapfig} %para acomodar figuras y que compartan espacio con texto
\usepackage{fancyhdr}
\pagestyle{fancy}
\fancyhf{}
\rfoot{Página \thepage}
\usepackage{enumerate}
\usepackage{cite}


% Enable SageTeX to run SageMath code right inside this LaTeX file.
% documentation: http://mirrors.ctan.org/macros/latex/contrib/sagetex/sagetexpackage.pdf
% \usepackage{sagetex}

\begin{document}

\begin{titlepage}%pagina inicial
\pagestyle{empty}%aún así no se pudo eliminar el número de página

\begin{figure}[h]
    \centering
    \includegraphics[width=8cm]{licfis}
    {\huge \maketitle}
    {\large Física Computacional\\Carlos Lizárraga Celaya}
\end{figure}
\end{titlepage}

%Contenido
\newpage
\tableofcontents

\newpage %introducción
\section{Intoducción}
En el siguiente texto se presentan los resultados y la información recabada en una investigación sobre la atmósfera. Se abordan dos temas principales, que son la estructura de la atmósfera y las propiedades que se analizan de ellas.

El motivo de este texto es proporcionar al lector información básica y procesable, para que así tenga interés por conocer más sobre lo que sucede en nuestro planeta, ya sea relacionada con la atmósfera terrestre o bien cualquier otro objeto de estudio científico.

El objetivo principal es comprender la estructura que hay en la atmósfera y las catarcerísticas que presentan. El cumplir con este objetivo nos sirve para generar interés en continuar con la investigación sobre la atmósfera.

Se presentan también las propiedades de la atmósfera, en esta parte de texto, se analizan los factores que más influyen en los fenómenos atmosféricos.

\newpage

\section{Estructura Atmosférica}
La atmósfera es una de las capas de materia que componen al planeta Tierra. Su estudio es importante ya que nuestra vida se desarrolla dentro de esta región terrestre. Dicha capa está constituida por gases, es por ello que se encuentra en el exterior de la tierra, iniciando desde la corteza terrestre y extendiéndose hasta.

La atmósfera se subdivide en, al menos, cuatro capas:

    \begin{itemize}
    \item Troposfera
    \item Estratosfera
    \item Mesosfera
    \item Termosfera
    \end{itemize}

Cada una de estas capas de la atmósfera poseen características  interesantes, las cuales nos ayudan a distinguir entre cada una de ellas, pues todas poseen propiedades físicas y químicas bastante distintivas.

Una de las funciones más importantes de la atmósfera es la de protegernos de las distintas formas de radiación provenientes del sol y del espacio, así como también de otras circunstancias que ponen en peligro la vida en la superficie.

Existe también otra región llamada exosfera; es la capa más externa de la atmósfera terrestre, hay quienes incluso no la consideran una capa de la atmósfera pues, parece más parte del espacio exterior que del planeta Tierra.

    \subsection{Tropósfera}
    En esta región de la atmósfera es en la que se desarrolla la vida terrestre; se extiende desde la superficie terrestre hasta alrededor de los 10km sobre esta, aproximadamente, su altura depende del lugar, adquiere una mayor altura en zonas cálidas que en las frías\cite{ncsu}. Contiene también tres cuartas partes de los gases que componen la atmósfera.
    
    La troposfera tiene efectos directos en nuestra vida, pues en ella se encuentra el aire que respiramos; de las propiedades que se presenten aquí, principalmente la temperatura podremos decidir qué tipo de clima se tendrá o para determinar  condiciones sobre las precipitaciones.
    

    La troposfera tiene un comportamiento térmico en el cual la temperatura decrece conforme se aumenta la altura con respecto a la superficie. Esto es porque la superficie terrestre absorbe y emite gran cantidad de calor, por eso el aire cercano a la superficie es más caliente que en las partes más alejadas.

    Al límite entre la troposfera y la estratosfera se le conoce como tropopausa, sobre ésta se encuentra la estratosfera.\cite{ncsu}

\subsection{Estratosfera}
La estratosfera es la segunda capa de la atmósfera terrestre. Contrario a lo que sucede en la troposfera, la temperatura aquí aumenta conforme se aumenta la altura. tiene un espesor de alrededor de los 40 km. 
    La estratosfera se encarga de albergar a la capa de ozono, que es cálida por ser la responsable de contener los rayos ultravioleta proveniente del sol.\cite{ncsu}
    
\subsection{Mesosfera}
La mesosfera es la tercera capa de la atmósfera, se encuentra sobre la estratosfera, el límite entre ambas regiones se denomina estratopausa. Se extiende desde los 50km de altura hasta los 85 km y la temperatura de esta capa disminuye conforme se aumenta la altura.

    Poco se conoce sobre esta capa de la atmósfera, pues la mayoría de los instrumentos que sirven para realizar mediciones atmosféricas no logran llegar a la mesosfera y los satélites orbitan sobre ella, por lo que es bastante complicado tener lecturas directas de ella. Existen, sin embargo, cohetes para realizar mediciones directas, pero debido a los costos y preparativos que requieren es bastante complicado realizar este tipo de mediciones de manera constante. \cite{ucar}.
    
    Se sabe que en esta capa se desintegran la mayoría de los meteoritos que entran en la atmósfera terrestre. Debido a esto se encuentran diversos metales suspendidos en el "aire" de la mesosfera. 
    
    Sobre la mesosfera se encuentra la termosfera, la frontera entre ambas capas se denomina mesopausa; en ella, debido a las turbulencias y corrientes, los gases que se encuentran en esta capa, y en las siguientes capas superiores, colisionan entre sí, provocando que sus moléculas se separen en los tipos elementos que los componen. \cite{ucar}.

\subsection{Termosfera}
Una de las más altas capas de la atmósfera es la termosfera, se encuentra sobre la mesofera. Es la capa más grande de la atmósfera, salvo por la exosfera si se considera a esta ultima como una capa de la atmosfera. 
    Se extiende desde los 90 km hasta entre los 500 y 1000 km, aproximadamente. Las temperaturas en la termosfera suben abruptamente, pero después se estabilizan, aumentando de manera equilibrada cuando se aumenta la altura.\cite{termo}.
    
    Esta región tiene tan poca densidad que hay ocasiones en las que se considera gran parte de la termosfera como espacio exterior, en esta capa se encuentran orbitando los satélites. Aquí sucede el fenómeno de la aurora boreal, que son visibles en los polos terrestres.
    Como en la mesopausa, en esta región los gases de descomponen en sus elementos que los componen, esto también gracias a la radiación proveniente del sol, como los rayos-X.\cite{termo}.
    
\newpage
\section{Propiedades de la Atmósfera}
Existen diversas propiedades tanto físicas, como químicas que definen a la atmósfera. Principalmente veremos las propiedades físicas, las cuales nos ayudan a entender y predecir los fenómenos que suceden en la atmósfera terrestre.

\begin{itemize}
\item \textbf{Temperatura:}

Este es uno de los factores más importantes de la atmósfera, pues nos permite saber la cantidad de energía que se tiene en la atmósferra. A partir de la temperatura es que suceden las otras propiedades, pues de ella dependen muchas características físicas que se presentan en la atmósfera \cite{temp}.

\item \textbf{Densidad}

La densidad depende de la temperatura del aire, es decir que está ligada a la situación de la temperatura en la atmósfera, ya que en lugares cálidos se tiene menos densidad que en los fríos, por la dilatación de los gases. La densidad nos permite saber la cantidad de masa por unidad de volumen, esto es que en la atmósfera nos permite conocer la cantidad de masa de los gases por cierta cantidad de volumen, para así especular la densidad del aire de una capa de la atmósfera por ejemplo.

La densidad de la atmósfera en la troposfera es mayor que en las demás capas, pues, es debido a la fuerza de gravedad que ejerce la tierra, hace que gran cantidad de los gases que conforman la atmósfera se vean atraídos hacia la superficie, siendo así que la concentración de gases en las demás capas es menor y por lo tanto son mucho menos densos que la troposfera; debido a esto es que se pueden tener globos para transportar personas o bien para realizar mediciones en la atmósfera y obtener datos importantes de ella \cite{den}.

\item \textbf{Presión}

La presión es otra propiedad que encontramos al estudiar la atmósfera. La densidad que es medida en la atmósfera es la fuerza ejercida sobre un área por el aire sobre un cuerpo. En la atmósfera es un indicador de la masa que se encuentra sobre ciertos lugares y ayuda a estudiar más sobre la cantidad de gases que se encuentran en la atmósfera y sus capas.

Al analizar la presión que ejerce el aire, nos ayuda a predecir si habrá precipitaciones, pues la presión es un factor importante para ello, tanto como la temperatura \cite{pres}.

\item \textbf{Humedad}

La humedad es otro factor importante, principalmente cuando se estudia la atmósfera para predecir fenómenos meteorológicos, pues la humedad nos cuenta la cantidad de vapor de agua que se encuentra en el aire.

Una forma de medir la humedad es mediante la humedad relativa, se utilizan porcentajes para establecer la cantidad de vapor de agua que se encuentra en el aire, obteniendo así la concentración de vapor que hay. La humedad relativa varía con la temperatura.

Otra forma de medir la humedad es con el punto de rocío, el cual nos dice la temperatura que se debe de alcanzar para que exista una concentración de vapor de agua saturada, o bien nos dice bajo que temperatura se consensará el vapor de agua que se encuentra en la atmósfera \cite{moist}
\end{itemize}
    
\newpage
\section{Conclusiones}
El estudio de la atmósfera es necesario, pues de ella depende la vida en la superficie. La atmósfera tiene propiedades difíciles de medir o establecer, pues existen factores que afectan y realizan cambios en el comportamiento de la atmósfera, es por ello que se complica determinar ciertas características atmosféricas. Si bien los datos que se obtienen con las herramientas necesarias son de ayuda, aún son diversos los factores que intervienen, además cada una de las propiedades depende de la otra directamente,todas están ligadas entre sí, por lo que cualquier cambio en una parte de la atmósfera se verá reflejado en otra parte.

La complejidad de lo que sucede en la atmósfera puede llegar a espantar a algunos, pues no es sencillo analizar y determinar un comportamiento como para predecir y entender todos los fenómenos que suceden en la atmósfera, incluso la formación de las nubes resulta ser un tema bastante complejo. 

Por todo lo anterior, debemos de mostrar interés en el desarrollo de herramientas físicas, computacionales, matemáticas e instrumentos que nos ayuden a estudiar más a fondo lo que sucede en nuestro planeta y sobre todo en la atmósfera, pues dependemos directamente de lo que en esta suceda.
    
\newpage
\begin{thebibliography}{9}
\bibitem{ncsu} 
Hall, Megan and McKemy, Dan and Lee, Amy and et. al.
\textit{The Atmosphere. Structure of the Atmosphere}. En linea. Visitada el 29 de enero del 2017:
\\\texttt{http://climate.ncsu.edu/edu/k12/.AtmStructure}
 
\bibitem{ucar} 
Russel, Randy.
\textit{The Mesosphere}. En linea. Visitada el 29 de enero del 2017:
\\\texttt{https://scied.ucar.edu/shortcontent/mesosphere-overview}
 
\bibitem{termo} 
Russel, Randy
\textit{The Thermosphere}. En linea. Visitada el 29 de enero del 2017:
\\\texttt{https://scied.ucar.edu/shortcontent/mesosphere-overview}

\bibitem{temp}
Hall, Megan and McKemy, Dan and Lee, Amy and et. al.
\textit{Atmospheric Properties.Temperature}. En linea. Visitada el 29 de enero del 2017:
\\\texttt{http://climate.ncsu.edu/edu/k12/.Temperature}

\bibitem{den}
Hall, Megan and McKemy, Dan and Lee, Amy and et. al.
\textit{Atmospheric Properties.Density}. En linea. Visitada el 29 de enero del 2017:
\\\texttt{http://climate.ncsu.edu/edu/k12/.density}

\bibitem{pres}
Hall, Megan and McKemy, Dan and Lee, Amy and et. al.
\textit{Atmospheric Properties.Pressure}. En linea. Visitada el 29 de enero del 2017:
\\\texttt{http://climate.ncsu.edu/edu/k12/.pressure}

\bibitem{moist}
Hall, Megan and McKemy, Dan and Lee, Amy and et. al.
\textit{Atmospheric Properties.Humidity}. En linea. Visitada el 29 de enero del 2017:
\\\texttt{http://climate.ncsu.edu/edu/k12/.humidity}

\end{thebibliography}



\end{document}
