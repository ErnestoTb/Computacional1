\documentclass[12pt]{article}
\usepackage[utf8]{inputenc}
\usepackage[T1]{fontenc}
\usepackage[spanish]{babel}%caracteres en español
\usepackage{verbatim}
\title{\huge \textbf{\textsc{Sondeos Atmosféricos\\ El Paso, Texas}}}%titulo en grande-negritas-versalitas
\author{J. Ernesto Torres Burruel}
\usepackage{graphicx}%para cargar imagenes
\graphicspath{{Imagenes/}}
\usepackage{wrapfig} %para acomodar figuras y que compartan espacio con texto
\usepackage{fancyhdr}
\pagestyle{fancy}
\fancyhf{}
\usepackage{enumerate}
\usepackage{cite}
\date{8 de febrero de 2017}
\usepackage{hyperref}
\usepackage{bookmark}
\fancyfoot[R]{Página \thepage}
\setlength\headheight{15 pt}
\fancyhead[L]{Jesús Ernesto Torres Burruel}
\fancyhead[R]{Física Computacional}
\usepackage{booktabs}
\usepackage[nottoc,numbib]{tocbibind}

\begin{document}

\begin{titlepage}%pagina inicial

\begin{figure}[h!]
    \centering
    \includegraphics[width=12cm]{licfis}
    \\[2 cm]
    \hrule
    \maketitle
    {\Large Física Computacional\\Carlos Lizárraga Celaya}
    \bigskip
    \hrule
    \thispagestyle{empty}
\end{figure}
 \begin{abstract}
    En este reporte se realiza la descripción para procesar datos archivos obtenidos de sondeos atmosféricos, proporcionados por una base de datos web. En este caso se utilizarán los datos recaudados por la estación de El Paso, Texas y y organizados por el Departamento de Ciencias Atmosféricas de la \textit{Universidad de Wyoming} en \url{http://weather.uwyo.edu/upperair/sounding.html}.
   
    \end{abstract} 

\end{titlepage}

\newpage

\tableofcontents
\setcounter{page}{2}
\newpage

\section{Introducción}
En esta práctica se utilizarán las herramientas para obtener y tratar datos de sondeos atmosféricos realizados en El Paso, Texas, utilizando GNUemacs y los comandos de la terminal, así como creando scripts para automatizar el proceso de obtención de datos y su transformación en archivos de texto con los cuales podremos trabajar para estudiar los datos obtenidos.

En el desarrollo de este texto se explicarán los procesos llevados a cabo durante la práctica para tratar los datos de un año de los sondeos realizados casi de manera diaria de manera que sea más fácil trabajarlos en un futuro.

Analizaremos también los datos de un día y comprobaremos que los datos otorgados por el sondeo de la estación muestra el comportamiento de la atmósfera de la misma manera en que se describe de fuentes bibliográficas.

\section{Sondeos atmosféricos}

Realizar estudios sobre la atmósfera nos ayuda a comprender más la estructura de nuestro planeta y a estudiar un factor que afecta nuestras vidas de manera diaria; es por esto que existen estaciones que se encargan de realizar estudios sobre la atmósfera de manera constante (dos veces al día principalmente), para ello realizan sondeos por medio del uso de globos de sondeo, que tienen instrumentos que realizan las mediciones de distintos parámetros, como: temperatura, presión atmosférica, altura, punto de rocío, punto de congelamiento, etc.

La información que recompilan los globos es enviada en tiempo real a la estación de sondeo, de esta manera la información es grabada para su posterior estudio. Para nosotros esta información está disponible desde el portal de la \textit{Universidad de Wyoming}, en \url{http://weather.uwyo.edu/upperair/sounding.html}.

\section{Uso de programas para trabajar con datos}
Trabajar con los datos y tablas producidas por mediciones de instrumentos muchas veces resulta difícil de analizar si se tiene una distribución poco aceptable o bien cuando existe una gran cantidad de texto que obstruye el análisis directo de los datos numéricos, como es el caso de descargar información recopilada y publicada por estaciones o centros de investigación, ya que la información la colocan en un formato con el cual alguien pueda guiarse para entender de qué trata cada dato, pero cuando queremos trabajar con ellos nos resulta inconveniente no tener la información fácil de extraer.

Las herramientas que nos ofrecen diversos programas para trabajar con archivos de distintas clases y textos es bastante útil al momento de tratar de obtener información de nuestro interés. En este caso se ven los procesos llevados a cabo para obtener información del interés de esta práctica. 

Los progrmas más comunes y que nos permiten trabajar de manera fácil con el contenido de un archivo con datos son la \textit{terminal de Linux}, el \textit{símbolo de sistema de windows (cmd)}, el editor de texto \textit{GNU emacs}, entre otros. También para analizar los datos se recurre a tablas de cálculo o programas graficadores, de esta forma se tiene una perspectiva más amplia de la información encontrada.

\subsection{Obtención de datos para un día}
Para obtener los datos de un día debemos acceder a la página \url{http://weather.uwyo.edu/upperair/sounding.html} e ingresar en la barra sobre el mapa los filtros para generar nuestros datos, en este caso se necesitan los datos de la estacion en El Paso, por lo que se ingresa el número de estación correspondiente (72364) y presionando la tecla enter se obtienen los datos para un día, ver figura \ref{fig:wyoming1}.


\begin{figure}[h]
\centering
\includegraphics[width= 10cm]{wyoming}

\caption{Página para obtención de datos de sondeos atmosféricos}
\label{fig:wyoming1}

\end{figure}

Una vez realizado lo anterior habremos obtenido una nueva ventana en nuestro navegador que contiene los datos del sondeo atmosférico de la estación elegida. Después solo es necesario copiar el texto que aparece en esta nueva ventana y pegarlo en un archivo de emacs. Se debe guardar el archivo pues nos servirá para realizar gráficas en otros programas.

Con los datos obtenidos para un día se realizó una gráfica para comprobar que se cumple lo que se dice sobre el comportamiento de la atmósfera. Es común saber que la presión atmosférica dsminuye conforme se aumenta la altura con respecto al suelo.  También sabemos que la temperatura disminuye con respecto a la altura en la troposfera, en la tropopausa hay temperatura casi constante y al inicio de la estratósfera, que es donde la mayoría de los globos de sondeo llega, la temperatura sube rápidamente.\cite{ncsu}

Utilizando gnuplot para graficar la presión contra la altura, de esta forma se posicionan puntos que se pueden aproximar de manera exponencial, cambiando la escala de la gráfica a una logarítmica podemos observar una recta, por lo que la presión con respecto a la altura tiene un comportamiento exponencial.

\begin{figure}[ht!]
\centering

\includegraphics[scale= 0.5]{press-alt}

\caption{Gráfica de presión contra altura.}

\label{fig:presal}

\end{figure}

\pagebreak

\begin{figure}[ht]
\centering

\includegraphics[scale= 0.5]{logpres}

\caption{Gráfica de presión contra altura en escala logarítmica.}

\label{fig:logp}

\end{figure}

Como se aprecia en la figura \ref{fig:logp}, los puntos se aproximan a una recta, esto quiere decir que existe un comportamiento exponencial para la variable de presión (eje y) como podrá consultar el lector en fuentes bibliográficas, como en la página web \emph{The Atmosphere. Structure of the Atmosphere} \cite{ncsu}

\begin{figure}[ht!]
\centering
\includegraphics[scale= 0.51]{tempaltura}
\caption{Gráfica de altura contra temperatura.}
\label{fig:tempal}
\end{figure}
En la figura \ref{fig:tempal} se encuentra un comportamiento de la temperatura conforme se aleja de la superficie, registrando que conforme sube su altura en la troposfera la temperatura disminuye, en la tropopausa (en este caso se encuentra entre los 15 km y los 20 km) permanece constante con respecto a la altura, por eso se forma una zona en la que los puntos ascienden verticalmente. Cuando sobrepasa la tropopausa la temperatura aumenta, esto quiere decir que el globo llegó a la estratósfera, en donde la temperatura aumenta con la altura.\cite{ncsu}

\subsection{Datos atmosféricos de sondeos realizados durante el 2016}

Estudiar los datos recaudados durante un día no es siempre la mejor manera de comprender cómo se comporta la atmósfera en la región en donde se realiza el sondeo. Una mejor manera de hacer análisis para hacer inferencias sobre la atmósfera es trabajando con observaciones de muchos sondeos, como por ejemplo un año. En este caso podríamos descargar de la página de la Universidad de Wyoming un mes a la vez, esto requeriría tiempo y si quisiéramos analizar varios lugares a la vez tendríamos que realizar este proceso varias veces, lo cual consumiría mucho esfuerzo y eso solo contando que descargamos datos de un solo año, probablemente necesitemos hacer una comparación de otros años para ver los cambios atmosféricos y climáticos que ha sufrido esa región.

Para eso haremos uso de un script que nos ahorrará el trabajo de descargar manualmente los archivos para un año de datos. Este script consiste en escribir en descargar en varios archivos los datos de cada mesa, estos estarán en un formato html, para consultar el script vaya al apéndice A \ref{fig:A}.

Para editar el script se hace uso del editor de texto GNU emacs, ya que posee las funciones necesarias para facilitarnos el proceso de edición de los datos; por ejemlpo, para sustiruir el número de estación que corresponde al lugar del que queremos obtener datos se hace uso de los comandos para sustituir palabras (o caracteres) en el texto, que es siguiendo los comandos: Esc-x query-replace, después se coloca  Una vez editado el script con nuestras especificaciones (lugar de origen de datos, año de los datos, etc.), para poder ejecutar el script se debe finalizar con .sh para poderlo correr desde nuestra terminal y efectuar las ordenes dadas.

Para poder pasar toda la información contenida en varios archivos a uno solo, la terminal (shell)será de gran ayuda, pues logra transcribir, buscar e identificar por medio de comandos lo que se contiene en los archivos creados por el script; para empezar se juntan todos los archivos en uno solo: \textit{cat arvchivo\textasteriskcentered \textgreater sondeos.txt}, en este nuevo archivo de texto permanecerá toda la infomración recabada por el script en los archivos de todo el año.

Si queremos saber la recurrencia de lanzamientos en el año debemos abrir el archivo con el comando cat sondeos.txt y buscar el renglón con la frase "Observaciones" con el comando grep, seguido del comando wc que cuenta renglones (\verb#cat sondeos.txt | grep Observaciones |wc# ). Se pueden hacer más especificaciones para encontrar palabras que nos ayuden a identificar los días o las horas, en este caso se usó  frases para encontrar cuántos días del año se realizaron sondeos (Ver \ref{apeb}. Apéndice B). Lo que se encontró fue:

\begin{table}[h]
\caption{\textbf{Días al año en que se realizaron sondeos}}
\label{fig:tabda}
\centering
\
\begin{tabular}{@{}|l|l|@{}}
\toprule
Hora & Días  \\ \midrule
00Z   & 362  \\
12Z  & 364 
\end{tabular}
\end{table}

\begin{table}[h]
\centering
\caption{Sondeos realizados por mes y hora}
\label{fig:somesho}
\begin{tabular}{|l|c|c|}
\textbf{Mes}        & \textbf{Días 00Z} & \textbf{Días 12Z} \\ \midrule
Enero      & 31       & 31       \\
Febrero    & 27       & 28       \\
Marzo      & 31       & 31       \\
Abril      & 30       & 30       \\
Mayo       & 31       & 31       \\
Junio      & 29       & 29       \\
Julio      & 31       & 31       \\
Agosto     & 30       & 31       \\
Septiembre & 30       & 30       \\
Octubre    & 31       & 31       \\
Noviembre  & 30       & 30       \\
Diciembre  & 31       & 31      
\end{tabular}
\end{table}

\section{Conclusiones}
El manejo de datos es importante para poder desarrollar ideas sobre el comportamiento de un fenómeno que se estudia, para tener más información sobre éste se debe de manejar una gran cantidad de información para poder tener un análisis más certero sobre el fenómeno, para así poder describirlo, predecirlo e incluso reproducirlo.

El manejo de datos a partir de las herramientas que nos brinda la tecnología, facilita y reduce el trabajo de un científico, haciendo que su desarrollo en la ciencia sea más productivo. Alrededor de un siglo atrás no se tenía la tecnología para procesar de manera automática, por lo que los científicos de esa y  otras épocas debían de trabajar datos de manera manual, por lo que tomaba gran parte de su tiempo encontrar datos y procesarlos.

En nuestros días debemos de estar preparados para manejar las herramientas tecnológicas que poseemos para reducir el tiempo de proceso de datos y facilitar nuestro trabajo.

\newpage

\begin{thebibliography}{9}
\bibitem{ncsu} 
Hall, Megan and McKemy, Dan and Lee, Amy and et. al.
\textit{The Atmosphere. Structure of the Atmosphere}. En linea. Visitada el 08 de febrero del 2017:
\\\texttt{http://climate.ncsu.edu/edu/k12/.AtmStructure}

\end{thebibliography}

\newpage

\section{Apéndice A: Script para descarga de datos}
\label{fig:A}

En este apéndice se incluye el script utilizado para descargar los datos de un año de sondeos atmosféricos realizados en la estación de El Paso, Texas. Los datos se obtienen gracias a la página de la \textit{Universidad de Wyoming}.

\begin{verbatim}
# Descarga por mes. Cambiar año de consulta. 
Ajustar el numero de estacion.
#!/bin/bash

# Despues de editar: chmod 755 script1.sh

# Para ejecutar: ./script1.sh

IFS=":"
LISTM31="01:03:05:07:08:10:12"
#LISTM31="01:03:05:07"
LISTM30="04:06:09:11"
#LISTM30="04:06"
LISTM28="02"

# Script para bajar info por mes. Año 2016, dentro del URL:
YEAR=2015

# Months 31 days

for i in $LISTM31 ; do

    /usr/bin/wget "http://weather.uwyo.edu/cgi-bin/sounding?
    region=naconf&TYPE=TEXT%3ALIST&YEAR=2016&MONTH
    =$i&FROM=0100&TO=3112&STNM=72364"

       /bin/sleep 5
done
# Months 30 days

for i in $LISTM30 ; do

    /usr/bin/wget "http://weather.uwyo.edu/cgi-bin/sounding?
    region=naconf&TYPE=TEXT%3ALIST&YEAR=2016&MONTH
    =$i&FROM=0100&TO=3012&STNM=72364"
      
      /bin/sleep 5
done
# Feb. 28 days

for i in $LISTM28 ; do
    /usr/bin/wget "http://weather.uwyo.edu/cgi-bin/sounding?
    region=naconf&TYPE=TEXT%3ALIST&YEAR=2016&MONTH
    =$i&FROM=0100&TO=2812&STNM=72364"
      
      /bin/sleep 5
 done
\end{verbatim}

\section{Apéndice B: Comandos utilizados}
\label{apeb}
En esta sección aparecen los comandos utilizados desde la terminal y en GNU emacs para procesar los archivos y datos.

\subsection{Terminal (shell)}

\verb+emacs script1.sh+ \hspace{1 cm}Con este comando emacs abre o crea (dependiendo si existe o no el archivo) para trabajar con el script para descargar datos.

\verb+which wget+ \hspace{1 cm} Con este comando se obtiene cuál es la ruta que posee la computadora para ingresar a internet (se debe introducir la ruta dada en el script1.sh)

\verb+ls -al+ \hspace{1 cm} Este comando nos permite ver los permisos que tiene nuestro archivo que contiene el script.

\verb+chmod script1.sh 755+ \hspace{1 cm} Con este comando le damos los permisos que requiere para que la computadora corra el script.

\verb+./script1.sh+ \hspace{1 cm} Este comando es para correr cualquier ejecutable en Linux desde la terminal. 

\pagebreak

Para juntar todos los archivos creados por el script1.sh en uno solo se usa el siguiente comando:
\begin{verbatim}cat sounding\?region\=naconf\&TYPE\=TEXT%3ALIST\&YEAR\=2016\&MO
NTH\=* > sondeos.txt
\end{verbatim}

\verb+ cat sondeos.txt | grep+ \textit{Palabra} \verb+| grep + \textit{Palabra} \verb+ | ...|wc+ \hfill \break
Con este comando se cuentan los renglones y muestra las palabras y caracteres de los renglones que contienen la(s) palabra(s) buscada con grep.

\subsection{Comandos en GNU emacs}

Una vez abierto el archivo script1.sh debemos realizar la búsqueda para sustituir según nuestras necesidades, para hacer esto usamos lo siguiente

\verb# Esc + <# (para localizarnos al inicio del archivo) \hfill \break
\verb# Esc + x query-replace + Enter# \textit{texto a sustituir} \verb# + Enter# \textit{texto por el que se sustituye} \verb#Enter#  \hfill \break
Para cambiar todas las ocurrencias debemos presionar la tecla \verb# ! # 

\verb# Ctrl + x + Ctrl + S# \hspace{1 cm} Para guardar los cambios en el archivo.

\verb#Ctrl + y# \hspace{1 cm} Para vaciar el contenido en el portapapeles.

\verb#Ctrl + x + Ctrl c# \hspace{1 cm} Para salir de GNU emacs.

\end{document}