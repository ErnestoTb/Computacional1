\documentclass{article}
\usepackage[utf8]{inputenc}
\usepackage[T1]{fontenc}
\usepackage[spanish]{babel}
\title{\huge \textbf{\textsc{Preguntas sobre la actividad 2}}}
\author{J. Ernesto Torres Burruel}
\usepackage{graphicx}
\graphicspath{{Imagenes/}}
\date{8 de Febrero de 2017}
\usepackage{enumerate}
\usepackage{fancyhdr}
\pagestyle{fancy}
\fancyhf{}
\fancyfoot[R]{Página \thepage}
\setlength\headheight{15 pt}
\fancyhead[L]{Jesús Ernesto Torres Burruel}
\fancyhead[R]{Física Computacional}

\begin{document}

\begin{titlepage}%pagina inicial

\begin{figure}[h!]
    \centering
    \includegraphics[width=12cm]{licfis}
    \\[2 cm]
    \hrule
    \maketitle
    {\Large Física Computacional\\Carlos Lizárraga Celaya}
    \bigskip
    \hrule
    \thispagestyle{empty}
\end{figure}

\end{titlepage}

\newpage

\section*{Preguntas}
Preguntas sobre la actividad:

\begin{enumerate}
\item{¿Cual es tu primera impresión del uso de bash/GNU emacs?}

La primera impresión que tuve de estos programas fue que tenía una visualización de texto bastante sencilla, pero esto es lo que se obtiene a simple vista. Al comenzar a usarlos me di cuenta de que muchos de los comandos utilizados en otros programas no tienen el mismo efecto en estos, como \verb#Ctrl + Z# en GNU emacs, también \verb#Ctrl + c# no está definido para copiar el texto seleccionado, se debe de hacer clic sobre el icono de copiar, aún no encuentro un comando del teclado con el cual copiar el texto.

\item{¿Ya lo habías utilizado?}

Lo había utilizado para la clase de Programación y lenguaje Fortran cursada en mis segundo semestre de la licenciatura en Física, pero los usos que le di fueron muy básicos y no utilicé casi ninguna de las funciones que estos ofrecen, ni siquiera la busqueda de palabras o su remplazamiento en los textos.

\item{¿Qué cosas se te dificultaron más en bash/Emacs?}

Adaptarme a los comandos para detener/cerrar los programas. Otra cosa fue realizar los comandos en el orden que debían hacerse para que el programa lograra interpretarlo, también el manejo de la visualización me resultó difícil cuando al presionar por accidente la sección de comandos ejecutandose en GNU emacs, ya que obstruye la visualización del texto en el que se trabaja y se pierde entre distintas ventanas que está ejecutando GNU emacs.

\item{¿Qué ventajas les ves a GNU emacs?}
Este editor de texto presenta muchas más opciones a los programadores/usuarios para poder controlar más el archivo en el cual están trabajando. Las funciones de GNU emacs permite facilitar el proceso de manejo de archivos y su creación.

\item{¿Qué es lo que mas te llamó la atención en el desarrollo de esta actividad?}

La gran cantidad de herramientas que podemos obtener para manejar datos a la hora de tratar de analizarlos.

\item{¿Qué cambiarías en esta actividad?}

Me gustaría haber tenido una guía para la creación de scripts y formar el mío, no solo editar uno ya hecho. Esto habría tomado más tiempo y creo que no es el objetivo del curso, tal vez haga falta una actividad especial para la creación de scripts.

\item{¿Que consideras que falta en esta actividad? }

Creo que es bastante completa porque se obtiene una introducción y una motivación al uso de estos programas para el manejo de información y edición de archivos. Así como la automatización de procesos.

\item{¿Puedes compartir alguna referencia nueva que consideras útil y no se haya contemplado?}

Para el manejo de estos programas no realicé ninguna consulta externa salvo la proporcionada en la actividad.

\item{¿Algún comentario adicional que desees compartir?}

GNU emacs y bash son bastante sencillos y consumen pocos recursos, por lo que el sistema de la computadora no se sobrecargará y se podrá tener un mayor desempeño por la máquina, se debe motivar más al uso de estos programas para el manejo de una gran cantidad de datos.

\end{enumerate}

\end{document}